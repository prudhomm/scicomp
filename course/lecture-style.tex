\usepackage[french]{babel}
\usepackage[latin1]{inputenc}
\usepackage[T1]{fontenc}
\usepackage{amsmath,amssymb,amsthm}
%\usefonttheme{professionalfonts}
\usepackage[colorlinks]{hyperref}

\mode<article>
{
  \usepackage{times}
  \usepackage{mathptmx}
  \usepackage[left=1.5cm,right=6cm,top=1.5cm,bottom=3cm]{geometry}

  
  \def\figwidth{.7}
}
\mode<presentation>{
  \def\figwidth{1.}
}
% \mode<presentation>
% {

% %\usetheme{Berlin}
% %\usetheme[secheader]{Boadilla}
% \usetheme{UJF}
% %\usetheme{Antibes}                  % Beamer theme v 3.0
% %\usecolortheme{lily}                % Beamer color theme
% \setbeamercovered{transparent}
% %%\includeonly{nonlineareq}
% }
%\usepackage{translator} % comment this, if not available
%\usepackage[xcolor=pst]{xcolor}
\usepackage{xcolor}
%\usepackage{pstricks}
\usepackage{pgfpages}
%\pgfpagesuselayout{1 on 1}[a4paper]
%\pgfpagesuselayout{4 on 1}[a4paper,border shrink=5mm]
\usepackage{subfigure}
\usepackage{colortbl}
\newcommand{\goodgap}{%
  \hspace{\subfigtopskip}%
  \hspace{\subfigbottomskip}}
\usepackage{xspace}
\usepackage[nofancy]{svninfo}
\svnInfo $Id: lecture.tex 64 2006-12-19 08:24:41Z prudhomm $

\pdfinfo {
  /Title    (Presentation)
  /Author   (Christophe Prud'homme <christophe.prudhomme@ujf-grenoble.fr>)
  /Keywords   (Numerical Methods, Scientific Computing)
  /ModifiedDate ( \svnInfoDate \svnInfoTime)
 }

\usepackage{multimedia,xmpmulti}
\DeclareGraphicsRule{*}{mps}{*}{}
\usepackage{graphicx}
\usepackage{tikz}
\usetikzlibrary{arrows,patterns,plotmarks,shapes,snakes,er,3d,automata,backgrounds,topaths,trees,petri,mindmap}

\usepackage[]{movie15}
\usepackage{filecontents,listings}
\definecolor{lbcolor}{rgb}{0.95,0.95,0.95}
\definecolor{cblue}{rgb}{0.,0.0,0.6}
\definecolor{lblue}{rgb}{0.1,0.1,0.4}
%%
%% Listing configuration for matlab/octave
%% Author: Christophe Prud'homme
%%
\renewcommand{\lstlistingname}{Program}
\lstset{language=Matlab,showspaces=false,showstringspaces=false,captionpos=t,literate={>>}{\ensuremath{>>}}1}
%\lstset{float}
\lstset{basicstyle=\small\ttfamily}
\lstset{lineskip=-2pt}
\lstset{keywordstyle=\color{red}\bfseries}
%\lstset{keywordstyle=\mdseries\color{red}}
\lstset{emph={inline},emphstyle=\color{red}\bfseries}
%\lstset{stringstyle=\ttfamily}
\lstset{commentstyle=\ttfamily\color{cblue}}
\lstset{backgroundcolor=\color{lbcolor},rulecolor=}
%\lstset{numbers=left}
%\lstset{numbers={none}}
%\lstset{numberstyle=\tiny}
%\lstset{numbersep=1pt}
\lstset{frame=single,framerule=0.5pt}
\lstset{belowskip=\smallskipamount}
\lstset{aboveskip=\smallskipamount}
\lstset{emph={inline,ones,global,whos,clear,save,load,fprintf,format,pause,title,
xlabel,ylabel,zlabel,axis,hold,clg,replot,polyfit,polyval}}


% Common settings for all lectures in this course

\def\lecturename{Calcul Scientifique en C++ et MPI}
\def\beamer@shortlecturename{SCICOMP}


\title{\insertlecture}

\author[C. Prud'homme]{Pr. Christophe Prud'homme\\
  \url{christophe.prudhomme@ujf-grenoble.fr}\\
  Dr. Mourad Ismail\\
  \url{ismail@ujf-grenoble.fr}\\
} 
\institute[UJF]{
  Universit� Joseph Fourier Grenoble 1
}

\date[2007-2008\xspace Rev: \svnInfoRevision]{2nd Semestre 2007-2008} 

\subject{Cours \lecturename}



%% Macros
\newcommand{\ddx}[1]{\ensuremath{\frac{\partial #1}{\partial x}}}
\newcommand{\ddy}[1]{\ensuremath{\frac{\partial #1}{\partial y}}}
\newcommand{\ddz}[1]{\ensuremath{\frac{\partial #1}{\partial z}}}
\newcommand{\dddxdx}[1]{\ensuremath{\frac{\partial^2 #1}{\partial x \partial x}}}
\newcommand{\dddxdy}[1]{\ensuremath{\frac{\partial^2 #1}{\partial x \partial y}}}
\newcommand{\dddydy}[1]{\ensuremath{\frac{\partial^2 #1}{\partial y \partial y}}}
\renewcommand{\div}{\operatorname{div}}
\newcommand{\rot}{\operatorname{rot}}

\newcommand{\uu}{\mathbf}
\newcommand{\ve}{\varepsilon}
\newcommand{\x}[1]{x^{(#1)}}

\newcommand{\N}{\ensuremath{\mathbb{N}}}
\newcommand{\Pol}{\ensuremath{\mathbb{P}}}
\newcommand{\pvec}[1]{\begin{pmatrix} #1\end{pmatrix}}
\newcommand{\vect}{\mathbf}
\newcommand{\p}{\partial}
\newcommand{\secQSS}[1]{\vspace{-0.6cm}\centerline{#1}\medskip}
\newcommand{\rr}{\mathbb{R}} %was \R
\newcommand{\cond}{K}%{K_2}
\newcommand{\pol}{\Pi}
\newcommand{\octave}{\textsc{Octave} }
\newcommand{\tableau}[1][{ }]{
    \rightline{#1\includegraphics[width=0.08\textwidth]{tableau}}
}
\newcommand{\verify}[1][{ }]{
  %%\hfill{#1\includegraphics[width=0.08\textwidth]{verify}}
  \hfill\square{}
}
\newcommand{\CommentLines}[1]{ }
\newcommand{\In}{\operatorname{in}}
\newcommand{\Out}{\operatorname{out}}


\newtheorem{proposition}{Proposition}

%% Lecture
%\AtBeginLecture{\frame{Le Cours d'Aujourd'hui:\\[1cm] \insertlecture}}
%\AtBeginSubsection[]{\frame{\frametitle{Outline}\tableofcontents[current]}}

% Beamer version theme settings
\definecolor{darkred}{rgb}{1.0,0.15,0.15}

\useoutertheme[height=0pt,width=2cm,right]{sidebar}
\usecolortheme{rose,sidebartab}
\useinnertheme{circles}
\usefonttheme[only large]{structurebold}

\setbeamercolor{sidebar right}{bg=black!15}
%\setbeamercolor{structure}{fg=darkred}
\setbeamercolor{structure}{fg=blue}
\setbeamercolor{author}{parent=structure}

\setbeamerfont{title}{series=\normalfont,size=\LARGE}
\setbeamerfont{title in sidebar}{series=\bfseries}
\setbeamerfont{author in sidebar}{series=\bfseries}
\setbeamerfont*{item}{series=}
\setbeamerfont{frametitle}{size=}
\setbeamerfont{block title}{size=\small}
\setbeamerfont{subtitle}{size=\normalsize,series=\normalfont}

\setbeamertemplate{navigation symbols}{}
\setbeamertemplate{bibliography item}[book]
\setbeamertemplate{sidebar right}
{
  {\usebeamerfont{title in sidebar}%
    \vskip1.5em%
    \hskip3pt%
    \usebeamercolor[fg]{title in sidebar}%
    \insertshorttitle[width=2cm-6pt,center,respectlinebreaks]\par%
    \vskip1.25em%
  }%
  {%
    \hskip3pt%
    \usebeamercolor[fg]{author in sidebar}%
    \usebeamerfont{author in sidebar}%
    \insertshortauthor[width=2cm-2pt,center,respectlinebreaks]\par%
    \vskip1.25em%
  }%
  \hbox to2cm{\hss\insertlogo\hss}
  \vskip1.25em%
  \insertverticalnavigation{2cm}%
  \vfill
  \hbox to 2cm{\hfill\usebeamerfont{subsection in
      sidebar}\strut\usebeamercolor[fg]{subsection in
      sidebar}\insertshortlecture.\insertframenumber\hskip5pt}%
  \vskip3pt%
}%

\setbeamertemplate{title page}
{
  \vbox{}
  \vskip1em
  {\huge Chapitre \insertshortlecture\par}
  {\usebeamercolor[fg]{title}\usebeamerfont{title}\inserttitle\par}%
  \ifx\insertsubtitle\@empty%
  \else%
    \vskip0.25em%
    {\usebeamerfont{subtitle}\usebeamercolor[fg]{subtitle}\insertsubtitle\par}%
  \fi%     
  \vskip1em\par
  Course \emph{\lecturename}\ , \insertdate\par
  \vskip0pt plus1filll
  \leftskip=0pt plus1fill\insertauthor\par
  \insertinstitute\vskip1em
}

\pgfdeclareimage[height=1cm]{logo}{../../../figures/ujf-logo-color-circle}
\logo{\pgfuseimage{logo}}

%%\logo{\includegraphics[width=2cm]{beamerexample-lecture-logo.pdf}}



% Article version layout settings

\mode<article>

\makeatletter
\def\@listI{\leftmargin\leftmargini
  \parsep 0pt
  \topsep 5\p@   \@plus3\p@ \@minus5\p@
  \itemsep0pt}
\let\@listi=\@listI


\setbeamertemplate{frametitle}{\paragraph*{\insertframetitle\
    \ \small\insertframesubtitle}\ \par
}
\setbeamertemplate{frame end}{%
  \marginpar{\scriptsize\hbox to 1cm{\sffamily%
      \hfill\strut\insertshortlecture.\insertframenumber}\hrule height .2pt}}
\setlength{\marginparwidth}{1cm}
\setlength{\marginparsep}{4.5cm}

\def\@maketitle{\makechapter}

\def\makechapter{
  \newpage
  \null
  \vskip 2em%
  {%
    \parindent=0pt
    \raggedright
    \sffamily
    \vskip8pt
    {\fontsize{36pt}{36pt}\selectfont Chapitre \insertshortlecture \par\vskip2pt}
    {\fontsize{24pt}{28pt}\selectfont \color{blue!50!black} \insertlecture\par\vskip4pt}
    {\Large\selectfont \color{blue!50!black} \insertsubtitle\par}
    \vskip10pt

    \normalsize\selectfont Printed version of the 
    course \emph{\lecturename}, \@date\par\vskip1.5em
    \hfill Christophe Prud'homme \& Mourad Ismail, Universit� Joseph Fourier Grenoble 1
  }
  \par
  \vskip 1.5em%
}

\let\origstartsection=\@startsection
\def\@startsection#1#2#3#4#5#6{%
  \origstartsection{#1}{#2}{#3}{#4}{#5}{#6\normalfont\sffamily\color{blue!50!black}\selectfont}}

\makeatother

\mode
<all>


%%% Local Variables:
%%% mode: latex
%%% TeX-PDF-mode: t
%%% TeX-parse-self: t
%%% x-symbol-8bits: nil
%%% TeX-auto-regexp-list: TeX-auto-full-regexp-list
%%% TeX-master: t
%%% ispell-local-dictionary: "french"
%%% End:

