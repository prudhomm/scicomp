% -*- coding: iso-latin-1 -*-
%
% SUMMARY:
% USAGE:
%
% AUTHOR:       Christophe Prud'homme
% ORG:          Christophe Prud'homme
% E-MAIL:       prudhomm@zion
%
% ORIG-DATE:  7-Apr-04 at 16:48:32
% LAST-MOD:  7-Apr-04 at 23:07:19 by Christophe Prud'homme
%
% DESCRIPTION:
% DESCRIP-END.

\date{January 21 2008}

\begin{document}

% For every picture that defines or uses external nodes, you'll have
% to apply the 'remember picture' style. To avoid some typing, we'll
% apply the style to all pictures.
\tikzstyle{every picture}+=[remember picture]
\tikzstyle{na} = [baseline=-.5ex]

%By default all math in TikZ nodes are set in inline mode. Change this to
% displaystyle so that we don't get small fractions.
\everymath{\displaystyle}

\lecture[4]{Linear Algebra libraries}{approx}
\subtitle{}

\begin{frame}
  \maketitle
\end{frame}

\begin{frame}
  \tableofcontents
\end{frame}


\section{Numerical Libraries}

\subsection{Some Libraries}
\begin{frame}{Some Basic/Standard Libraries}
  \begin{block}{}
    \begin{itemize}
    \item Blas/Lapack/Scalapack (atlas,goto blas)
      \url{}
    \item Arpack/Arpack++
      \url{}
    \item fftw3
      \url{}
    \item SuiteSparse
      \url{}
    \item Superlu
      \url{}
    \item gmm
      \url{}
    \item PETSc
      \url{}
    \item Trilinos
      \url{}
    \end{itemize}

  \end{block}
  \begin{alertblock}{Questions}
    Find their respective website \\
    Give a 5 lines maximum description of these libraries
  \end{alertblock}
\end{frame}

\subsection{What Do They Have in Common?}
\begin{frame}<1-2>{What Do They Have in Common?}
  \begin{block}{What Do They Have in Common?}
    \only<2>{Framework or subroutines for Linear Algebra or computing intensive tasks}
  \end{block}
\end{frame}

%%\againframe{scprog}

\begin{frame}{What do these Libraries contain?}
  \begin{block}{What do these Libraries contain?}
    \begin{itemize}
    \item Algebraic Containers :  vectors, matrices, multi-D arrays
    \item Algorithms (LU, SVD, CG, GMRES,...)
    \item Utility functions (View, save, transform  ...)
    \end{itemize}
  \end{block}
\end{frame}

\subsection{Blitz++}
\begin{frame}
  \frametitle{Blitz++: Features}
  \begin{block}{\url{http://www.oonumerics.org/blitz/}}
    \begin{itemize}
    \item Provide an efficient array data structure based on template meta-programming techniques
    \item Provide Indices data structure to facilite array entries access
    \item Provide an easy enough interface
    \end{itemize}
  \end{block}
  \begin{alertblock}{Questions}
    Study \url{http://www.oonumerics.org/blitz/} \\
    Describe with your own words what Blitz could do for you (max 1 page)
  \end{alertblock}

\end{frame}

\begin{frame}{Fetch, Compile and Install Blitz}
  \begin{block}{Fetch, Compile and Install Blitz}
  \begin{itemize}
  \item Get the latest version available (which one is it?)
  \item Compile it (\alert{hint:} use \lstinline!configure! and \lstinline!make!)
    what options of \lstinline!configure! do you think could be useful?
  \item Install it in \lstinline!$HOME! directory, say
    \lstinline!$HOME/blitz! (\alert{hint:} use \lstinline!make install prefix=$HOME/blitz!)
  \end{itemize}
  \end{block}
  \begin{alertblock}{Questions}
    Write down the exact commands you used
  \end{alertblock}
\end{frame}

\begin{frame}[fragile]
  \frametitle{Simple Blitz++ Sample Code}

  \begin{filecontents*}{programs/blitz-matlab.m}
    N=10 ;
    A = zeros(10,10)
    A = ones(10,10)
    A = eyes(10,10)
    A(N/2,:) = -1
    A(1:3,N/2) = -2
    A'
  \end{filecontents*}

  \begin{alertblock}{Question: Blitz++ versus Matlab}
    Write a code using Blitz++
    that would replicate the following Octave commands:
  \end{alertblock}
  \lstinputlisting[language=octave,emph={ones,eyes}]{programs/blitz-matlab.m}


\end{frame}

\begin{frame}[fragile]
  \frametitle{Solve the Heat Equation using Blitz++}

  \begin{alertblock}{Questions}
    Write a code using Blitz++ to solve the following equation in 2D
    on $\Omega = ] 0, 1 [ \times ]0, 1 [$, find $u(x,t)$ such that
    \begin{align}
      \label{eq:1}
      \displaystyle \frac{\partial u}{\partial t} - \Delta u &= 1 \mbox{ on } \Omega\\
      u(\cdot,t) &= 0 \mbox{ on } \partial \Omega,\ \forall t \geq 0 \\
      u(x,0) &= 0  \mbox{ on }  \Omega
    \end{align}
    Write the code in vectorial form, save results in column format
    and use octave or gnuplot to visualize the isovalues of $u(x,t)$
    when it reaches a steady state
  \end{alertblock}

\end{frame}

\begin{frame}{Similar Libraries and Projects}

  \begin{block}{Similar Libraries and Projects}
    \begin{itemize}
    \item Boost.UBlas
    \item Boost.MultiArray
    \item Glas
    \item Gmm
    \end{itemize}
  \end{block}
  \begin{alertblock}{Questions}
    Find their website\\
    Describe briefly what they can do
  \end{alertblock}
\end{frame}
% \section{Some Useful Programs}


% \begin{frame}[fragile]
%   \frametitle{The Laplacian in nD, n=1,2,3 -- }

%   \lstinputlisting[firstline=262,lastline=334]{programs/laplacian.cpp}
% \end{frame}

%%\lecture{\centerline{\Large Gmm++ et PETSc}}{semaine 2}


\section{Programming Environment}
\subsection{C/C++/Fortran}
\begin{frame}[containsverbatim]{C++ Compiler}
  There are mainly three stages when compiling a code
  \begin{enumerate}
  \item Preprocessing : all lines starting by \lstinline!#! are
    processed (e.g. inclusion of header file)
  \item Compiling : transform C++ files into Objects files
  \item Linking : Link objects files together with libraries to form
    the executable
  \end{enumerate}
\begin{lstlisting}{language=ksh}
g++ -I$HOME/blitz/include -o myprog myprog.cpp \
 -lblitz -L$HOME/blitz/lib
\end{lstlisting}
  \begin{alertblock}{}
    \lstinline!man! is your friend, Try also Google
  \end{alertblock}
\end{frame}

\subsection{Build Systems}
\begin{frame}[containsverbatim]{Makefile}
  A Makefile simplifies the build process of a code
\begin{lstlisting}{language=makefile}
CXX=g++
CXXFLAGS=-O2
CPPFLAGS=-I$HOME/blitz/include -I$HOME/gmm/include
LDFLAGS=-L$HOME/blitz/lib
LIBS=-lblitz -lblas
PROGRAMS=myprog
all: $(PROGRAMS)
# link
$(PROGRAMS):
	$(CXX) -o $@ $@.o $(LDFLAGS) $(LIBS)
# compile
.cpp.o:
	$(CXX) $(CPPFLAGS) $(CXXFLAGS)  -c $*.cpp
# clean the directory
clean:
	-rm -f *.o $(TARGETS)
# dependencies
myprog:      myprog.o

.SUFFIXES: .cpp .o .hpp
\end{lstlisting}
\end{frame}

\begin{frame}{Other Build Systems}
  \begin{itemize}
  \item autotools: autoconf,automake,libtool (sh+m4+perl)
  \item scons (python based)
  \item cmake
  \item bjam
  \item ...
  \end{itemize}
\end{frame}


\section[Linear Algebra]{Librairies C/C++ d'Alg�bre Lin�aire}
\subsection[Gmm++]{Gmm++}
\begin{frame}{: Matrices Creuses et Solveurs}
  \begin{itemize}
  \item Un ensemble de commande g�n�riques (clear, clean, scalar product, scale, norms, ...)
  \item Structures de donn�es pour vecteurs et matrices creuses
  \item Op�rations vecteurs-vecteurs de formats diff�rents (sparse, dense, skyline)
  \item Multiplication Matrix-Vector de format diff�rents (sparse, dense, skyline, row major, column major, ...)
  \item Solveurs lin�aires g�n�riques (cg, bicgstag, qmr, gmres ...) avec pr�conditionneurs pour matrices creuses (ILUT, ILUTP, ILDLT, ...) - issu de la librairie ITL
  \item Sous-matrices
  \item Factorisation LU et QR pour les matrices denses
  \item Calcul de valeurs propres pour les matrices denses
  \end{itemize}
\end{frame}

\begin{frame}{Web site}
  \url{http://www-gmm.insa-toulouse.fr/getfem/gmm_intro}
\end{frame}

\begin{frame}[containsverbatim]{Utilisation Basique}
\begin{lstlisting}
gmm::dense_matrix<double> M(3, 3);
gmm::clear(M); // M = 0.
M(0,0) = M(1,1) = M(2,2) = 2.0; // M = 2 * Id.
M(1,2) = 1.0;
std::vector<double> X(3), B(3), Bagain;

// B = [1 2 3]
B[0] = 1.0; B[1] = 2.0; B[2] = 3.0;
gmm::lu_solve(M, X, B);
gmm::mult(M, X, Bagain);
std::cout << M << " times " << X
          << " is equal to " << Bagain << std::endl;
\end{lstlisting}

\end{frame}

\begin{frame}[containsverbatim]{Dans le cadre d'une EDP}
\begin{lstlisting}
// number of degrees of freedom.
int nbdof = 1000;
// a sparse matrix
typedef gmm::rsvector<double> rsvector_t;
typedef gmm::row_matrix< rsvector_t > csrmat_t;
csrmat_t M(nbdof, nbdof);
// Unknown and left hand side.
std::vector<double> X(nbdof), B(nbdof);

... here the assembly of the pde discretization ...
... stiffness matrix ...
... and left hand side ...
// computation of a preconditioner (ILUT)

gmm::ilut_precond< csrmat_t > P(M, 10, 1e-4);
// defines an iteration object,
// with a max residu of 1E-8
gmm::iteration iter(1E-8);
// execute the GMRES algorithm
gmm::gmres(M, X, B, P, 50, iter);
std::cout << "The result " << X << std::endl;
\end{lstlisting}
\end{frame}

\begin{frame}{Le Stockage Creux}
  \begin{equation*}
    A=
    \begin{pmatrix}
       1 & 2 & 0 & 0 & 10\\
       0 & 5 & 0 & 7 & 0\\
       2 & 0 & 1 & 3 & 0\\
       0 & 0 & 1 & 2 & 3\\
       0 & 0 & 0 & 0 & 12
    \end{pmatrix}
  \end{equation*}
\end{frame}

\begin{frame}[containsverbatim]{Les Vecteurs}
\begin{lstlisting}
// le vecteur standard de gmm
std::vector<T>
// write sparse, optimis�e pour l'�criture!
gmm::wsvector<T>
// read sparse, optimis�e pour la lecture!
gmm::rsvector<T>
\end{lstlisting}
\end{frame}

\begin{frame}[containsverbatim]{Gmm++: Les Matrices}
\begin{lstlisting}
// dense row matrix
gmm::row_matrix< std::vector<double> > M1(10, 10);

// sparse column matrix
gmm::col_matrix< gmm::wsvector<double> > M2(5, 20);

// Stockage FORTRAN (column major)
gmm::dense_matrix<T>

// compressed sparse row
gmm::csr_matrix<T>

// compressed sparse column
gmm::csc_matrix<T>
\end{lstlisting}
\end{frame}


\begin{frame}[containsverbatim]{Assemblage et Stockage}
\begin{lstlisting}
typedef gmm::wsvector<double> wsvector_t;
// rem : non contigue en memoire
gmm::row_matrix< wsvector_t > M1;
...
assembly operation on M1
...
M1(i,j) = b;
...
// contigue en memoire
// interfacable avec FORTRAN/UMFPACK/SUPERLU
gmm::csc_matrix<double> M2;

gmm::clean(M1, 1E-12);
gmm::copy(M1, M2);
\end{lstlisting}
\end{frame}

\begin{frame}[containsverbatim]{Quelques M�thodes}
\begin{lstlisting}
std::cout << V << std::endl;
std::cout << M << std::endl;
// set to zero all the components
gmm::clear(V);

// set to zero all the components
gmm::clear(M);

// set to zero all the components
gmm::clean(V, 1E-10);

// whose modulus is less or
// idem for a matrix M.
gmm::clean(M, 1E-10);

// transpose
gmm::transposed(M)
\end{lstlisting}
\end{frame}

\begin{frame}[containsverbatim]{Quelques Op�rations}
\begin{lstlisting}
typedef std::vector<double> vector_t;

// V * 10.0 ---> V
gmm::scale(V, 10.0);
vector_t V1(10);
gmm::wsvector<double> V2(10);
gmm::clear(V1);
...
// V1 + V2 --> V2
gmm::add(V1, V2);
cout << V2;

gmm::row_matrix< vector_t > M1(10, 10);
...
// M1 * V2 --> V1
gmm::mult(M1, V2, V1);
\end{lstlisting}
  \begin{alertblock}{}
    Ces op�rations sont optimis�es
  \end{alertblock}
\end{frame}

\begin{frame}[containsverbatim]{Les Normes}
\begin{lstlisting}
// sum of the modulus of the components of vector V.
gmm::vect_norm1(V)
// Euclidean norm of vector V.
gmm::vect_norm2(V)
// Euclidean distance between V1 and V2.
gmm::vect_dist2(V1, V2)
// infinity norm of vector V.
gmm::vect_inf(V)
// Euclidean norm of matrix M
// (called also Frobenius norm).
gmm::mat_euclidean_norm(M)
// Max norm (defined as max(|m_ij|; i,j ))
gmm::mat_normmax(M)
// max(sum(|m_ij|, i), j)
gmm::mat_norm1(M)
// max(sum(|m_ij|, j), i)
gmm::mat_norminf(M)
\end{lstlisting}
\end{frame}

\begin{frame}[containsverbatim]{Factorisation LU Dense}
\begin{lstlisting}
// compute the LU factorization of M in M. ipvt should
// be an std::vector<size_t> (of size
// gmm::mat_nrows(M)) which will contain the indices
// of the pivots.
gmm::lu_factor(M, ipvt)

// solve the system LUx = b. LU is the LU
// factorization which has to be computed first.
gmm::lu_solve(LU, ipvt, x, b)

// solve the system Mx=b calling the lu factorization
// on a copy of M.
gmm::lu_solve(M, x, b)

// invert A calling the LU factorization and the latter
// procedure.
gmm::lu_inverse(A)

// compute the determinant of A calling the LU
// factorization and the latter function.
gmm::lu_det(A)
\end{lstlisting}
\end{frame}

\begin{frame}[containsverbatim]{Factorisation LU Creuse}
  \begin{alertblock}{}
    On utilise SuperLU
    \begin{itemize}
    \item \lstinline!CPPFLAGS=-DGMM_USES_SUPERLU!
    \item \lstinline!CXXFLAGS=-I/usr/include/superlu!
    \item \lstinline!LIBS=$(LIBS) -lsuperlu -lblas!
    \end{itemize}
  \end{alertblock}
\begin{lstlisting}
typedef gmm::wsvector<double> wsvector_t;
gmm::row_matrix< wsvector_t > Dt(ndof,ndof);
assembly operation on Dt
gmm::csc_matrix<double> D(ndof,ndof);
gmm::clean(Dt, 1E-12); gmm::copy(Dt, D);
// same for F
std::vector<double> F(ndof);
std::vector<double> U(ndof);

double condest;
gmm::SuperLU_solve(D, U, F, condest );
std::cout << "condition number estimation "
          << condest << "\n";
\end{lstlisting}
\end{frame}

\begin{frame}[containsverbatim]{M�thodes It�ratives}
\begin{lstlisting}
// The matrix
typedef std::vector<double> vector_t;
gmm::row_matrix<vector_t> A(10, 10);
// Right hand side and Unknown
std::vector<double> B(10), X(10);
// Optional scalar product for cg
gmm::identity_matrix PS;
// Optional preconditioner
gmm::identity_matrix PR;
// Iteration object with the max residu
gmm::iteration iter(10E-9);
// restart parameter for GMRES
size_t restart = 50;
// Conjugate gradient
gmm::cg(A, X, B, PS, PR, iter);
// BICGSTAB BiConjugate Gradient Stabilized
gmm::bicgstab(A, X, B, PR, iter);
// GMRES generalized minimum residual
gmm::gmres(A, X, B, PR, restart, iter)
// Quasi-Minimal Residual method.
gmm::qmr(A, X, B, PR, iter)
\end{lstlisting}
\end{frame}


\begin{frame}[containsverbatim]{Pr�conditionneurs}
\begin{lstlisting}
// No preconditioner
gmm::identity_matrix P;

// diagonal preconditioner
gmm::diagonal_precond<matrix_type> P(SM);

// incomplete (level 0) ldlt preconditioner.
// Fast to be computed but less efficient than
// gmm::ildltt_precond.
gmm::ildlt_precond<matrix_type> P(SM);

// incomplete ldlt with k fill-in and threshold
// preconditioner. Efficient but could be costly.
gmm::ildltt_precond<matrix_type> P(SM, k, threshold);

// incomplete (level 0) ilu preconditioner.
// Very fast to be computed but less efficient
// than gmm::ilut_precond
gmm::ilu_precond<matrix_type> P(SM);
\end{lstlisting}
\end{frame}

\begin{frame}{Fetch, Compile and Install Gmm++}
  \begin{block}{Fetch, Compile and Install Gmm++}
  \begin{itemize}
  \item Get the latest version available (which one is it?)
  \item Compile it (\alert{hint:} use \lstinline!configure! and \lstinline!make!)
    what options of \lstinline!configure! do you think could be useful?
  \item Install it in \lstinline!$HOME! directory, say
    \lstinline!$HOME/blitz! (\alert{hint:} use \lstinline!make install prefix=$HOME/blitz!)
  \end{itemize}
  \end{block}
  \begin{alertblock}{Questions}
    Write down the exact commands you used
  \end{alertblock}
\end{frame}

\begin{frame}[fragile]
  \frametitle{Simple Gmm++ Sample Code}

  \begin{filecontents*}{programs/blitz-matlab.m}
    N=10 ;
    A = zeros(10,10)
    A = ones(10,10)
    A = eyes(10,10)
    A(N/2,:) = -1
    A(1:3,N/2) = -2
    A'
  \end{filecontents*}

  \begin{alertblock}{Question: Gmm++ versus Matlab}
    Write a code using Gmm++
    that would replicate the following Octave commands:
  \end{alertblock}
  \lstinputlisting[language=octave,emph={ones,eyes}]{programs/blitz-matlab.m}


\end{frame}

\begin{frame}[fragile]
  \frametitle{Solve the Heat Equation using Gmm++}

  \begin{alertblock}{Questions}
    Write a code using Gmm++ to solve the following equation in 2D
    on $\Omega = ] 0, 1 [ \times ]0, 1 [$, find $u(x,t)$ such that
    \begin{align}
      \label{eq:1}
      \displaystyle \frac{\partial u}{\partial t} - \Delta u &= 1 \mbox{ on } \Omega\\
      u(\cdot,t) &= 0 \mbox{ on } \partial \Omega,\ \forall t \geq 0 \\
      u(x,0) &= 0  \mbox{ on }  \Omega
    \end{align}
    Write the code, save results in column format and use octave or
    gnuplot to visualize the isovalues of $u(x,t)$ when it reaches a
    steady state
  \end{alertblock}

\end{frame}

\subsection{PETSc}

\begin{frame}[containsverbatim]{Petsc: Intro}
  \begin{itemize}
  \item ``Portable, Extensible Toolkit for Scientific Computation''
  \item Librairie C pour la r�solution de pbs aux EDPs
  \item PETSc s'appuie sur MPI pour le parall�lisme
  \item Approche orient�e objet :
    Voir \url{http://www-unix.mcs.anl.gov/petsc/petsc-2/}
  \end{itemize}
\end{frame}

\begin{frame}[containsverbatim]{Petsc: Contenu}
  \begin{itemize}
  \item Manipulation de structures de donn�es pour matrices creuses
  \item R�pertoire de solveurs de syst�mes lin�aires et de pr�conditionneurs
  \item Solveurs non lin�aires
  \item R�pertoire de m�thodes d'int�gration num�rique de syst�mes diff�rentiels
  \item Outils de calcul parall�le haut niveau reposant sur MPI
  \end{itemize}
\end{frame}


\begin{frame}[containsverbatim]{PETSc: Hi�rarchie}

\begin{tikzpicture}[node distance=1.2cm,scale=.2]
      \tikzstyle{every entity}=[draw=red!80,fill=red!40,thick]
      \tikzstyle{level 2}=[sibling distance=10mm,
      set style={{every node}+=[fill=white]}]


      \node[entity] (matrice) []  {\tiny
        \begin{tabular}[c]{cccccc}
          \multicolumn{6}{c}{\textbf{Matrices}}\\
          CSR  & Block CSR & Block Diagonal & Dense & Matrix-Free & Others\\
          (AIJ) & (BAIJ) & (BDIAG) & & &
        \end{tabular}
      };
      \node[entity] (precond) [above of=matrice]  {\tiny
        \begin{tabular}[c]{cccccc}
          \multicolumn{6}{c}{\textbf{Preconditioners}}\\
          Additive Schwarz & Block Jacobi & Jacobi & ILU & ICC & Others
        \end{tabular}};
      \node[entity] (solver) [above of=precond]  {\tiny
        \begin{tabular}[c]{cccccccc}
          \multicolumn{8}{c}{\textbf{Solvers}}\\
          GMRES & CG & CGS & BICGSTAB & TFQMR & Richardson & Chebychev & Others
        \end{tabular}

      };
      \node[entity] (time) [above right of=solver,yshift=.5cm,xshift=3cm]  {\tiny
        \begin{tabular}[c]{cccc}
          \multicolumn{4}{c}{\textbf{Time Steppers}}\\
          Euler & Backward Euler & Pseudo Time & Others\\
          & &  Stepping &
        \end{tabular}

      };
      \node[entity] (nlsolver) [left of=time,xshift=-5cm]  {\tiny
        \begin{tabular}[c]{cc}
          \multicolumn{2}{c}{\textbf{NonLinear Solvers}}\\
          \begin{tabular}[c]{c|c}
            \multicolumn{2}{c}{Newton Based}\\\hline
            Line Search & Trust Region
          \end{tabular} & Others

        \end{tabular}

      };

      \node[entity] (distarray) [below left of=matrice,xshift=-3cm,yshift=-.5cm]  {\tiny\textbf{Distributed Array}};
      \node[entity] (index)    [below right of=distarray,xshift=5cm,yshift=.5cm] {\tiny
        \begin{tabular}[c]{cccc}
          \multicolumn{4}{c}{\textbf{Index Sets}}\\
          Indices & Block Indices & Stride & Others
        \end{tabular}
      };
      \node[entity] (array)   [left of=index,xshift=-3cm,yshift=-.5cm]                  {\tiny\textbf{Array}};
  \end{tikzpicture}
\end{frame}
\begin{frame}[containsverbatim]{Tutoriel}
  The PETSc website proposes some tutorials


  \url{http://www-unix.mcs.anl.gov/petsc/petsc-as/documentation/exercises/index.html}
\end{frame}
\begin{frame}
  \frametitle{PETSc: Exercises}

  \begin{alertblock}{Exercises}
    \begin{enumerate}
    \item  Compile the first exercise (0). Write the commands.
    \item Run the first exercise(0). Write the output.
    \end{enumerate}
  \end{alertblock}
\end{frame}


\end{document}
\subsection{Newton}
\label{sec:newton}

\begin{frame}{}

\end{frame}

\section[Lagrange]{Lagrange basis}
\label{sec:lagrange-basis}

\subsection{Interpolation}
\label{sec:interpolation}


\begin{frame}{}

\end{frame}

\subsection{Runge Phenomenon}
\label{sec:runge-phenomenon}

\begin{frame}{}

\end{frame}

\subsection{Chebyshev polynomials}
\label{sec:chebysh-polyn}

\begin{frame}{}

\end{frame}

\subsection[Interval]{Interval Approximation}
\label{sec:interv-interp}

\begin{frame}{}

\end{frame}


\section{Multidimension approximation}
\label{sec:mult-interp}

\subsection{Basic geometry}
\label{sec:basic-geometry}

\begin{frame}{}

\end{frame}

\subsection{Coordinate systems}
\label{sec:coordinate-systems}

\begin{frame}{}

\end{frame}

\subsection{Primal basis}
\label{sec:primal-basis}

\begin{frame}{}

\end{frame}

\subsection{Polynomials and set of polynomials}
\label{sec:polyn-set-polyn}

\begin{frame}{}

\end{frame}

\subsection{Functionals and sets of functionals}
\label{sec:functionals}

\begin{frame}{}

\end{frame}

\subsection{Finite elements}
\label{sec:finite-elements}

\begin{frame}{}

\end{frame}

\subsection{Geometric Transformation}
\label{sec:geom-transf}

\begin{frame}{}

\end{frame}

\subsection{Curvilinear domain}
\label{sec:curvilinear-domain}

\begin{frame}{}

\end{frame}


\section{Other Representations}
\label{sec:other-repr}

\subsection{Fourier}
\label{sec:fourier-polynomials}

\begin{frame}{}

\end{frame}


\subsection{Wavelets}
\label{sec:wavelets}

\begin{frame}{}

\end{frame}

\section{MPI}
\label{sec:mpi}

\subsection{Send/Recv}
\label{sec:sendrecv}

\begin{frame}{}

\end{frame}

\section{Programming}
\label{sec:programming}

\subsection{MPI Send/Recv}
\label{sec:mpi-sendrecv}

\begin{frame}{}

\end{frame}

\subsection{C++}
\label{sec:c++}

\begin{frame}{}

\end{frame}

\subsection{Libraries}
\label{sec:libraries}

\begin{frame}{}

\end{frame}



\end{document}


%%% Local Variables:
%%% mode: latex
%%% TeX-master: "scicomp-libalg-print"
%%% TeX-PDF-mode: t
%%% TeX-parse-self: t
%%% x-symbol-8bits: nil
%%% TeX-auto-regexp-list: TeX-auto-full-regexp-list
%%% ispell-local-dictionary: "american"
%%% End:

