%
%
% SUMMARY:
% USAGE:
%
% AUTHOR:       Christophe Prud'homme
% ORG:          Christophe Prud'homme
% E-MAIL:       prudhomm@zion
%
% ORIG-DATE:  7-Apr-04 at 16:48:32
% LAST-MOD:  7-Apr-04 at 23:07:19 by Christophe Prud'homme
%
% DESCRIPTION:
% DESCRIP-END.

\date{January 14,21 2008}

\begin{document}

% For every picture that defines or uses external nodes, you'll have
% to apply the 'remember picture' style. To avoid some typing, we'll
% apply the style to all pictures.
\tikzstyle{every picture}+=[remember picture]
\tikzstyle{na} = [baseline=-.5ex]

%By default all math in TikZ nodes are set in inline mode. Change this to
% displaystyle so that we don't get small fractions.
\everymath{\displaystyle}

\lecture[2]{Galerkin method}{approx}
\subtitle{}

\begin{frame}
  \maketitle
\end{frame}

\begin{frame}
  \tableofcontents
\end{frame}

\section[Weighted Residuals]{Method of Weighted Residuals (MWR)}
\label{sec:meth-weight-resid}

\subsection{Goals}
\label{sec:goals}

\begin{frame}{Goals}
  \begin{itemize}
  \item approximate exact solution numerically
  \item statisfies a finite number of conditions (i.e. not everywhere in the domain region)
  \item choices of conditions to satisfy:
    \begin{itemize}
    \item type of numerical method
    \item type of projection operator
    \end{itemize}
  \end{itemize}
  \begin{example}
    In the \textbf{collocation} method, the PDE is satisfied at a few
    distincts points rather than at every points of the domain
    region
  \end{example}
\end{frame}

\subsection{Formulation}
\label{sec:formulation}

\begin{frame}{Abstract formulation}
  To begin, denote $\mathbb{L}$ a linear differential operator defined on a domain $\Omega$ and $u$ the solution of
  \begin{equation}
    \label{eq:1}
    \mathbb{L} (u) = 0
  \end{equation}
  including some initial conditions (IC) and boundary conditions (BC).

  Assume that we have $u$ approximated accurately by $u^\delta$ such that
  \begin{equation}
    \label{eq:2}
    u \stackrel{\text{accurately}}{\approx}  u^{\delta }(x, t ) = u_0(x,t) + \sum_{i=1}^{N_{\text{dof}}}\ \hat{u}_i(t) \Phi_i(x)
  \end{equation}
  where
  \begin{itemize}
  \item $u_0(x,t)$ satisfies IC and BC
  \item $\Phi_i(x)$ are called \emph{trial} functions or \emph{expansion} functions, they also  satisfy IC and \emph{homogeneous} BC
  \item $\hat{u}_i(t)$ are called the \emph{degrees of freedom} (DOF)
  \end{itemize}
\end{frame}

\begin{frame}{Equation on $u^\delta$}
  We now plug \eqref{eq:2} into \eqref{eq:1}, we get
  \begin{equation}
    \label{eq:3}
    \mathbb{L}(u^{\delta}) = R( u^{\delta} )
  \end{equation}
  where $R$ is a non-zero \emph{residual}.

  We still have no unique way of determining the $(\hat{u}_i(t))$. We
  shall then further restrict $R$ which will allow
  \begin{itemize}
  \item to reduce \eqref{eq:3} to a system of ordinary differential equations in $\hat{u}_i(t)$
  \item if \eqref{eq:1} is independant of time, to determine $\hat{u}_i$ directly from a system of algebraic equations
  \end{itemize}
\end{frame}

\begin{frame}{Restrictions on $R$}
  First introduce the Lebesgue inner product $(f,g)$ defined as
  \begin{equation}
    \label{eq:4}
    (\ f,\ g\ )\ =\ \int_\Omega \ f(\mathbf{x})\ g(\mathbf{x})\ d\mathbf{x}
  \end{equation}
  To further restrict $R$, we ask $R$ to be ``orthogonal'' to a set of
  \emph{test}(\emph{weight}) functions $v_j$ for
  $j=1...N_{\text{dof}}$ with respect to $(\cdot, \cdot)$:
  \begin{equation}
    \label{eq:5}
    \int_\Omega\ v_j(\mathbf{x})\ R  = 0 \quad j = 1 ... N_{\text{dof}}
  \end{equation}
  The weighted residual is then said to be zero. Hence the name of the method.

  If \eqref{eq:5} is satisfied then we have from \eqref{eq:3} that
  \begin{equation}
    \label{eq:58}
    (v_j, \mathbb{L}(u^\delta) ) = 0, \quad j=1...N_{\text{dof}}
  \end{equation}
\end{frame}

\subsection{Type of methods}
\label{sec:type-methods}

\begin{frame}{Type of method}
  As $N_{\text{dof}} \rightarrow 0$, $R \rightarrow 0$ and
  $u^{\delta}(x,t) \rightarrow u(x,t)$.

  The nature of the scheme is determined by
  \begin{itemize}
  \item the choice of the expansion functions $\Phi_i(\mathbf{x})$
  \item the choice of the weights functions $v_j(\mathbf{x})$
  \end{itemize}

  \begin{table}[H]
    \centering

    \begin{tabular}[c]{ll}
    \rowcolor[gray]{1}
    Test/Weights & Type of method\\

    \rowcolor[gray]{.7}
    $v_j(\mathbf{x}) = \delta(\mathbf{x} - \mathbf{x_j})$ 	& Collocation \\


    \rowcolor[gray]{1}
    $v_j(\mathbf{x})=
      \begin{cases}
        1 & \text{inside} \ \Omega_j\\
        0 & \text{outside} \ \Omega_j
      \end{cases}$ & Finite volume(subdomain)\\

    \rowcolor[gray]{.7}
    $v_j(\mathbf{x}) = \frac{\partial R}{\partial \hat{u}_j}$ & Least square\\

    \rowcolor[gray]{1}
    $v_j(\mathbf{x}) = \Phi_j(\mathbf{x})$ & Galerkin\\

    \rowcolor[gray]{.7}
    $v_j(\mathbf{x}) = \Psi_j(\mathbf{x}) (\neq \Phi_j(\mathbf{x}))$ & Petrov-Galerkin\\
    \end{tabular}
    \caption{Table of the type of methods}
    \label{tab:1}
  \end{table}

\end{frame}

\begin{frame}{Collocation}
  The test function $v_j$ is the Dirac delta function computed at a point $\mathbf{x}_j$.
  At a collocation point $\mathbf{x}_j$, we have
  \begin{equation}
    \label{eq:6}
    ( v_j(\mathbf{x}), R ) = ( \delta(\mathbf{x} - \mathbf{x}_j), R ) =  R(\mathbf{x}_j) = 0
  \end{equation}
  Thus \eqref{eq:1} is satisfied by $u^{\delta}$ at $\mathbf{x}_j$
\end{frame}

\begin{frame}{Finite volume/subdomain}
  \begin{itemize}
  \item split the domain $\Omega$ into $N_{\text{dof}}$ subdomains
    $\Omega_j$ and defined $v_j$ as in table~\ref{tab:1}.

    \begin{equation}
      \label{eq:7}
      v_j(\mathbf{x})=
      \begin{cases}
        1 & \text{inside} \ \Omega_j\\
        0 & \text{outside} \ \Omega_j
      \end{cases}
    \end{equation}
  \item very popular method in aerodynamics
  \item can be seen as a technique to recover a conservation statement from a partial differential equation (PDE).
  \end{itemize}
\end{frame}

\begin{frame}{Least-Square}
  \begin{itemize}
  \item originate from least-square estimation by Gauss
  \item the choice of $v_j$ allows to determine $\hat{u}_i$ which minimse $(R,R)$
  \item this formulation has increased popularity lately when used in the context of spectral hp element methods
  \end{itemize}
\end{frame}

\begin{frame}{Galerkin}
  \begin{itemize}
  \item also known as Bubnov/Galerkin
  \item the test functions are the same as the trial functions, i.e. $v_j = \Phi_j$
  \item an extension is the Petrov-Galerkin method (aka generalized Galerkin method)
    where the test functions are usually a perturbation of the trial
    functions to improve numerical stability (e.g. SUPG)
  \end{itemize}

\end{frame}

\begin{frame}{Our focus}
  We shall focus on \textbf{Galerkin} methods \emph{continuous} and \emph{discontinuous}

  \begin{block}{Remark on MWR}
    MWR illustrates \emph{how} to construct different types of
    numerical techniques and the projection operator to use.

    It does not define the \emph{type of expansion} or \emph{approximation spaces}
  \end{block}

  \begin{itemize}
  \item spectral: global expansion functions
  \item finite : local expansion functions in small/finite regions
  \end{itemize}
\end{frame}

\section{Galerkin Methods}
\label{sec:galerkin-methods}

\subsection{Formulation}
\label{sec:formulation-1}

\begin{frame}{A Generic example and applications}
  Consider that
  \begin{equation}
    \label{eq:8}
    \mathbb{L}(u) \equiv \nabla^2 u + f = 0
  \end{equation}


  The applications of this mathematical model are quite broad
  \begin{itemize}
  \item irrotational fluid
  \item steady state heat equation
  \item electrical potential
  \item gravitational potential
  \end{itemize}

In 1D we get
\begin{equation}
  \label{eq:9}
  \mathbb{L}(u) \equiv \frac{\partial^2 u}{\partial x^2} + f = 0
\end{equation}
\end{frame}

\begin{frame}{Boundary conditions}
  \begin{itemize}
  \item \emph{Dirichlet} or \emph{essential} boundary conditions
    \begin{equation}
      \label{eq:10}
      u(0) = g_{\mathcal{D}}
    \end{equation}
    usually enforced \emph{explicitely} (i.e. strongly)
  \item \emph{Neumann} or \emph{natural} boundary conditions
    \begin{equation}
      \label{eq:11}
      \frac{\partial u}{\partial x}(1) = g_{\mathcal{N}}
    \end{equation}
    usually enforced \emph{implicitely} (e.g. weakly). In 2/3 D that
    would be $\nabla u \cdot \mathbf{N} = g_{\mathcal{N}}$ where $\mathbf{N}$ is the outward normal  on $\partial \Omega$.
  \end{itemize}
\end{frame}

\begin{frame}{Weak form}
  We multiply~\eqref{eq:8} by  a test function $v$ and integrate to get
  \begin{equation}
    \label{eq:12}
    (v, \mathbb{L}(u))\ =\ \int_0^1 v\ \Big( \frac{\partial^2 u}{\partial x^2} + f \Big)\ =\ 0
  \end{equation}
  Applying, integration by parts (or Gauss theorem), we then get
  \begin{equation}
    \label{eq:13}
    \int_0^1 \frac{\partial u}{\partial x}\frac{\partial v}{\partial x}\ dx\ =\  \int_0^1\ v\ f dx + \Big[ v\ \frac{\partial u}{\partial x} \Big]_0^1
  \end{equation}
  Recalling that, for example, $v(0) = 0$ and $\frac{\partial u(1)}{\partial x} = g_{\mathcal{N}}$ we have
  \begin{equation}
    \label{eq:14}
    \int_0^1 \frac{\partial u}{\partial x}\frac{\partial v}{\partial x}\ dx\ =\  \int_0^1\ v\ f dx + v(1)\ g_{\mathcal{N}}
  \end{equation}
\end{frame}
\begin{frame}{Discretisation}
  We now replace $v$ usually by a finite expansion $v^{\delta}$ and likewise for $u$ to get
  \begin{equation}
    \label{eq:16}
    \int_0^1 \frac{\partial u^{\delta}}{\partial x}\frac{\partial v^{\delta}}{\partial x}\ dx\ =\  \int_0^1\ v^{\delta}\ f dx + v^{\delta}(1)\ g_{\mathcal{N}}
  \end{equation}
\end{frame}

\begin{frame}{Mixed or Robin boundary conditions}
  We can also have mixed Dirichlet and Neumann boundary conditions known as Robin boundary conditions:
  \begin{equation}
    \label{eq:17}
    \alpha\ \nabla u \cdot \mathbf{N }\ +\ \beta\ u\ =\ g_{\mathcal{R}}
  \end{equation}
  where $\alpha \neq 0$ and $\alpha, \beta, g_{\mathcal{R}}$ known.
  To impose this condition, we express for example in 1D
  \begin{equation}
    \label{eq:18}
    \frac{\partial u}{\partial x}(1) = \frac{1}{\alpha}( g_{\mathcal{R}} - \beta u(1) )
  \end{equation}
  to get
  \begin{equation}
    \label{eq:19}
    \int_0^1 \frac{\partial u^{\delta}}{\partial x}\frac{\partial v^{\delta}}{\partial x}\ dx\ +\ \frac{\beta}{\alpha} v^{\delta}(1)\ u^{\delta}\ =\  \int_0^1\ v^{\delta}\ f dx + \frac{v^{\delta}(1)\ g_{\mathcal{R}}}{\alpha}
  \end{equation}

  \begin{block}{Remark}
    This boundary condition has application for example to treat
    convective heat transfer.
  \end{block}
\end{frame}

\subsection{Mathematical Formulation}
\label{sec:math-form}

\begin{frame}{Helmholtz equation}
  Consider now
  \begin{equation}
    \label{eq:15}
    \mathbb{L}(u) \equiv \frac{\partial^2 u}{\partial x^2} - \lambda u + f = 0,\ u(0) = g_{\mathcal{D}},\ \frac{\partial u}{\partial x}(1) = g_{\mathcal{N}}
  \end{equation}
  with $\lambda > 0$, and $0 < x < l$.

  If $u(x)$ and $v(x)$ are sufficiently smooth and we do integration by parts (Gauss theorem) and get
  \begin{equation}
    \label{eq:20}
    \int_0^l\ \frac{\partial u}{\partial x}\ \frac{\partial v}{\partial x} + \int_0^l\ \lambda\ v\ u - \int_0^l v f dx - \Big[ v\frac{\partial u}{\partial x}\Big]_0^l = 0
  \end{equation}
  Denote then
  \begin{equation}
    \label{eq:21}
    \begin{array}[c]{rl}
      a(v, u) &= \int_0^1 \frac{\partial u}{\partial x}\ \frac{\partial v}{\partial x} + \lambda\ v\ u\\
      f(v) &= \int_0^l vf + \Big[ v\frac{\partial u}{\partial x}\Big]_0^l
    \end{array}
  \end{equation}
  We solve for
  \begin{equation}
    \label{eq:22}
    a(v,u) = f(v)
  \end{equation}
\end{frame}

\begin{frame}{Energy space and norm}
  In structural mechanics $a(v,u)$ is called the \emph{strain energy} and the space of all functions with \emph{finite} strain on $\Omega$ is the energy space:
  \begin{equation}
    \label{eq:23}
    E(\Omega) = \{u | a(u,u) < \infty \}
  \end{equation}
  Usually we note $||u||_E = \sqrt{a(u,u)}$. The functions in $E$ are
  said to be $H^1$ functions ie $\int u^2 < \infty$ and $\int \nabla^2 u <
  \infty$.
\end{frame}

\begin{frame}{Test and trial spaces}
  The trial space  is denoted $\mathcal{X}$
  \begin{equation}
    \label{eq:24}
    \mathcal{X} = \{ u | u \in H^1,\ u(0) = g_{\mathcal{D}}\}
  \end{equation}

  The test space  is denoted $\mathcal{V}$
  \begin{equation}
    \label{eq:25}
    \mathcal{V} = \{ u | u \in H^1,\ u(0) = 0 \}
  \end{equation}

  $\mathcal{V}$ is also said to be $H^1_0$ (``0'' refers to the fact we have an homogeneous space )
\end{frame}

\begin{frame}{Formulation}
  Find $u \in \mathcal{X}$ such that
  \begin{equation}
    \label{eq:26}
    a(v, u) = f(v)\ \forall v \in \mathcal{V}
  \end{equation}
  which is infinite dimensional. We  discretize it and denote
  $\mathcal{X}^{\delta} \subset \mathcal{X}$ and $\mathcal{V}^{\delta}
  \subset \mathcal{V}$.

  \begin{block}{Remark}
    $\mathcal{X}^\delta$ and $\mathcal{V}^\delta$ are said to be conforming.
  \end{block}

  \begin{block}{Remark}
  In spectral/hp element methods we have two type of methods: \emph{h
    type} (element size) and \emph{p type} (polynomial
  order). $\delta$ refers to both and is a function of $h$ or/and $p$.
  \end{block}

  Find $u^\delta \in \mathcal{X}^\delta$ such that
  \begin{equation}
    \label{eq:27}
    a(v^\delta, u^\delta) = f(v^\delta)\ \forall\ v^\delta \in \mathcal{V}^\delta
  \end{equation}
\end{frame}

\begin{frame}{Dirichlet BC}
  To impose Dirichlet BC, we \emph{lift} the solution by decomposing $u^\delta \in \mathcal{X}^\delta$  into
  \begin{equation}
    \label{eq:28}
    u^\delta = \underbrace{u^{\mathcal{H}}}_{\text{homogeneous} \in {\mathcal{V}^\delta}} + \underbrace{u^{\mathcal{D}}}_{\text{Dirichlet} \in {\mathcal{X}^\delta}}, u^{\mathcal{H}} = 0, u^{\mathcal{D}} = g_{\mathcal{D}}
  \end{equation}
  We solve then for:

  Find $u^\delta = u^{\mathcal{H}} + u^{\mathcal{D}}$ where $u^{\mathcal{H}} \in \mathcal{V}^\delta$ and $u^{\mathcal{D}} \in \mathcal{X}^\delta$ such that
  \begin{equation}
    \label{eq:29}
    a( v^\delta, u^{\mathcal{H}}) = f^*(v^\delta), \ \forall v^\delta \in \mathcal{V}^\delta
  \end{equation}
  with
  \begin{equation}
    \label{eq:30}
    f^*(v^\delta) = f(v^\delta) - a(v^\delta, u^{\mathcal{D}})
  \end{equation}
\end{frame}

\begin{frame}{Some math properties}

  \begin{itemize}
  \item $a$ is a symmetric bilinear form.
  \item $a$ is continuous (bounded)
    \begin{equation}
      \label{eq:31}
      \exists\ C_1 > 0\ \text{ such that } |a(v,u)| \leq C_1 ||u||_1\ ||v||_1,\quad C_1 < \infty,\ \forall u, v
    \end{equation}
  \item $a$ is elliptic(coercive)
    \begin{equation}
      \label{eq:32}
      \exists\ C_2 > 0 \text{ such that } a(u,u)| \geq  C_2 ||u||^2_1,\quad C_2 < \infty,\ \forall u
    \end{equation}
\end{itemize}

\begin{block}{Lax Milgram}
  With the above properties and thanks to Lax-Milgram, \eqref{eq:29} has a \emph{unique} solution.
\end{block}
\end{frame}

\begin{frame}{Other properties}
  \begin{itemize}
  \item the error $\epsilon = u-u^\delta$ is \emph{orthogonal} to all function in $\mathcal{V}^\delta$
    \begin{equation}
      \label{eq:33}
      a(v^\delta, \epsilon) = 0\ \forall\ v^\delta\ \in\ \mathcal{V}^\delta
    \end{equation}
    This is called the \textbf{Galerkin orthogonality}.
  \item Minimal property of the error in the enery norm
    \begin{equation}
      \label{eq:34}
      ||u-u^\delta||_E = \mathrm{min}_{w^{\delta} \in \mathcal{V}^\delta}\ ||u-w^\delta||_E
    \end{equation}
  \end{itemize}
\end{frame}

\begin{frame}{Equivalence of polynomial expansion}
  A trivial observation of Galerkin approximation.

  Consider two linearly independent expansions which span
  $\mathcal{X}^\delta$ and $u_1$, $u_2$  solutions of \eqref{eq:29} in these
  expansions
  \begin{equation}
    \label{eq:36}
    u_1^\delta(x) = \sum_{i=1}^P \alpha_i \Phi_i(x), \quad u_2^\delta(x) = \sum_{i=1}^P \beta_i \Psi_i(x)
  \end{equation}

  where $P$ is the same polynomial degree ( $\Phi_i$ and $\Psi_i$ are in $\mathbb{P}^P$).

  Uniqueness implies that
  \begin{equation}
    \label{eq:37}
    u_1^\delta(x) = u_2^\delta(x) \Rightarrow \sum_{i=1}^P \alpha_i \Phi_i(x) = \sum_{i=1}^P \beta_i \Psi_i(x)
  \end{equation}
  meaning that any \textbf{error estimates} are independent of the type of expansion and depend only on $P$ (degree).

  \begin{alertblock}{Remark}
    There are nonetheless differences for example in the numerical
    conditioning of the algebraic system. The choice of the expansion
    is in fact very important.
  \end{alertblock}
\end{frame}

\section{1D expansion basis}
\label{sec:1d-expansion-basis}

\begin{frame}{Local/Global Strategy}
  We have ``defined'' the FE framework in terms of Galerkin formulation.

  We can now consider different types of expansions bases. Here is the scheme
  \begin{itemize}
  \item standard elemental regions $\rightarrow$ defines standard expansion bases
  \item assembly global expansion from local ones
  \end{itemize}

  This is an efficient scheme for spectral/hp element methods. Once we
  know how to numerically integrate/differentiate numerical
  functions.

  \begin{alertblock}{}
    This is also valid un multi-dimensional expansion
  \end{alertblock}
\end{frame}

\begin{frame}{Polynomial expansions}
  We are interested here in polynomial expansions (we could also
  consider Fourier/Wavelets).  This comes historically from the Taylor
  expansion that allows to write a function $f$ locally in terms of
  polynomials.

  \begin{block}{Advantage}
    This allows for discrete integration rules (and also
    differentiation) which enables easy computer implementation
  \end{block}
\end{frame}

\begin{frame}{$h$ type expansion}
  We set a fixed polynomial order order. Convergence is achieved by
  reducing the element size ($h$ is typically the characteristic size
  of the element).

  \begin{block}{Advantages}
    Geometric flexibility especially in high dimensions
  \end{block}
\end{frame}

\begin{frame}{$p$ type expansion}
  We set a fixed mesh size. Convergence is achieved by
  increasing the polynomial order in each element.

  \begin{block}{Advantages}
    rapid convergence for \textbf{smooth} problems.
  \end{block}
  When we have only one domain the $p$ type method is called a \textbf{spectral method}.
\end{frame}

\begin{frame}{$hp$ type}
  Combines both the $h$ and $p$ type method and their advantages. Of
  course this is much more complicated to handle.
\end{frame}


\subsection{Elemental decomposition}

\begin{frame}{Domain splitting}
  We decompose the domain $\Omega$ into a set of non-overlapping domains $\Omega^e$, ie
  \begin{equation}
    \label{eq:38}
    \Omega = \cup_{e=1}^{N_{\text{el}}}\ \Omega^e, \quad \cap_{e=1}^{N_\text{el}}\ \Omega^e = \varnothing
  \end{equation}
  So if we have
  \begin{equation}
    \label{eq:39}
    \Omega = \{\ x\ | \ 0 < x < \ell \}
  \end{equation}
  Introducing $0 < x_1 < ...< x_{N_\text{el}-1}  < x_{N_\text{el}} = \ell$, we have
  \begin{equation}
    \label{eq:40}
    \Omega^e = \{ x\ |\ x_{e-1} < x < x_e \}
  \end{equation}
  \begin{alertblock}{Complexity}
    If we have $N_{\text{el}}$ elements and $N_{\text{dof}}$ degrees
    of freedom as $N_{\text{el}}$ increases, the number of expansion
    functions increases. This is not economical if we are to store the $\Phi_i$.
  \end{alertblock}
  We shall introduce a standard element domain where most operations will be done.
  \begin{equation}
    \label{eq:41}
    \Omega^{\text{st}} = \{ \xi\ | \ -1 \leq \xi \leq 1 \}
  \end{equation}
\end{frame}
\begin{frame}{Local/global mapping}
  Consider $\phi_0$ and $\Phi_1$ two expansion functions defined in $\Omega^{\text{st}}$ such that
  \begin{equation}
    \label{eq:42}
    \phi_0(\xi) =
    \begin{cases}
      \frac{1-\xi}{2}, \ & \xi \in \Omega^{\text{st}}\\
      0, & \xi \notin \Omega^{\text{st}}\\
    \end{cases},
    \quad
    \phi_1(\xi) =
    \begin{cases}
      \frac{1+\xi}{2}, \ & \xi \in \Omega^{\text{st}}\\
      0, & \xi \notin \Omega^{\text{st}}\\
    \end{cases}.
  \end{equation}
  we then map $\Omega^{\text{st}}$ onto $\Omega^e$ via $\chi^e(\xi)$ defined by
  \begin{equation}
    \label{eq:43}
    x = \chi^e(\xi) = \frac{1-\xi}{2} x_{e-1} + \frac{1+\xi}{2} x_e, \quad \xi \in \Omega^{\text{st}}
  \end{equation}
  The inverse of the mapping is then
  \begin{equation}
    \label{eq:44}
    \xi = \Big[\chi^e\Big]^{-1}(x) = 2 \frac{x - x_{e-1}}{x_e-x_{e-1}} -1 , \quad x \in \Omega^{e}
  \end{equation}
\end{frame}
\begin{frame}{Local/Global expansions}
  We then have $\Phi_i(x)$ represented by the mapping of $\phi_p$ onto $\Omega^{\text{st}}$. For example we can write $\Phi_0$ and $\Phi_1$

  \begin{equation}
    \label{eq:57}
    \begin{array}[c]{l}
      \Phi_0(x) =
    \begin{cases}
      \frac{x-x_1}{x_0-x_1}, & x \in \Omega^1\\
      0 & x \notin \Omega^1
    \end{cases}
    =
    \begin{cases}
      \phi_0(\xi) = \phi_0(\Big[ \chi^1\Big]^{-1} (x)), & x \in \Omega^1\\
      0 & x \notin \Omega^1
    \end{cases}\\
    \Phi_1(x) =
    \begin{cases}
      \frac{x-x_1}{x_1-x_0}, & x \in \Omega^1\\
      \frac{x-x_2}{x_1-x_2}, & x \in \Omega^2\\
      0 & \text{otherwise}
    \end{cases}
    =
    \begin{cases}
      \phi_1(\xi) = \phi_1(\Big[ \chi^1\Big]^{-1} (x)), & x \in \Omega^1\\
      \phi_0(\xi) = \phi_0(\Big[ \chi^1\Big]^{-2} (x)), & x \in \Omega^2\\
      0 & \text{otherwise}
    \end{cases}\\
    \end{array}
  \end{equation}

  \begin{block}{Remark}
    The inverse mapping might not be analytic (explicit). It would
    depend on the parametric mapping used. This is particularly true
    in multiple dimension when using curved domains.
  \end{block}


\end{frame}


\begin{frame}{Parametric mapping}
  The transformation $\chi^e(\xi)$ maps the local coordinates
  $\xi$($\xi \in \Omega^{\text{st}}$) to the global coordinates $x$,
  $x \in \Omega^e$.

  \begin{block}{Interpretation}
    It can be interpreted as expanding the global coordinate $x$ using a linear (finite) expansion as follows
    \begin{equation}
      \label{eq:35}
      x = \chi^e(\xi) = \phi_0(\xi) x_{e-1} + \phi_1(\xi) x_e, \quad \xi \in \Omega^{\text{st}}
    \end{equation}
  \end{block}

  This technique is called \emph{parameteric mapping}, it is
  \begin{itemize}
  \item \emph{iso-parametric} if $\chi^e(\xi)$ is of the same degree as the local expansions
  \item \emph{sub/super-parametric} if $\chi^e(\xi)$ is of lower resp. greater polynomial degree than the local expansion
  \end{itemize}
  \begin{block}{Remark}
    This is particularly useful for \emph{curved} domains
  \end{block}
\end{frame}
\begin{frame}{Remark}
  The transformation in \eqref{eq:35} is linear as well as its inverse.

  Hence $\phi_p(\Big[\chi^e\Big]^{-1} (x))$ is a polynomial in $x$ as
  well as in $\xi$. Hence the global expansions of $u^{\delta}$ are
  polynomial in $x$.

  \begin{alertblock}{High order polynomials and curved geometry}
    In the context of high order polynomials and curved geometries,
    this may not remain true. In particular, the global expansion may
    not be polynomial in $x$. However the local expansion remains
    polynomial in $\xi$.
  \end{alertblock}
\end{frame}


\subsection{Global assembly}
\label{sec:glonbal-assembly}

\begin{frame}{Global assembly}
  The local to global expansion suggests/implies the concept of \textbf{global
    assembly} (or direct stiffness summation): we shall compute
  certain local expansion and assemble them to obtain the global expansion

  In 1D (we will follow the same construction in multi-D), we start with expanding $u^\delta(x)$
  \begin{equation}
    \label{eq:46}
    u^\delta(x) = \sum_{i=0}^{N_\text{dof}}\ \hat{u}_i \Phi_i(x)
  \end{equation}
  \eqref{eq:46} then writes using the parametric mappings $\chi^e, e=1...N_{\text{el}}$
  \begin{equation}
    \label{eq:45}
    u^\delta(x) = \sum_{i=0}^{N_\text{dof}}\ \hat{u}_i \Phi_i(x) = \sum_{e=0}^{N_\text{el}}\ \sum_{p=0}^{P}\ \hat{u}^e_p \phi_p(\xi)
  \end{equation}
  where $P$ is the polynomial order and $\phi_p(\xi) = \phi_p( \Big[ \chi^e\Big]^{-1} (x) )$.

  \begin{block}{Remark}
    There are more $u^e_p$ than $\hat{u}_i$. We require further conditions
  \end{block}
\end{frame}

\begin{frame}{Extra condition}
  we shall require that the expansion is $C^0$. If $P=1$, $N_{\text{el}} = 3$ and $N_{\text{dof}} = 4$, this leads to
  \begin{equation}
    \label{eq:47}
    \begin{array}[c]{rcl}
      \hat{u}^1_1 &=& \hat{u}^2_0 \\
      \hat{u}^2_1 &=& \hat{u}^3_0
    \end{array}
  \end{equation}
  We have  then the local global correspondance
  \begin{equation}
    \label{eq:48}
    \begin{array}[c]{rcl}
      \hat{u}^1_0 &=& \hat{u}_0 \\
      \hat{u}^1_1 &=& \hat{u}^2_0 = \hat{u}_1\\
      \hat{u}^2_1 &=& \hat{u}^3_0  = \hat{u}_2\\
      \hat{u}^3_1 &=& \hat{u}_3 \\
    \end{array}
  \end{equation}
\end{frame}


\begin{frame}{Global/local mapping}
  Now denote $\hat{u}_g$ the vector of global coefficients in the global expansion
  \begin{equation}
    \label{eq:49}
    \hat{u}_g^T = \big[ \hat{u}_0, ..., \hat{u}_{N_{\text{dof}}-1} \big]^T
  \end{equation}

  Denote $\hat{u}^e$ the vector of local coefficients in the element $e$,
  \begin{equation}
    \label{eq:50}
    \hat{u}^e = \big[ \hat{u}^e_0, \hat{u}^e_1 \big]^T
  \end{equation}
  and finally $\hat{u}^\ell$ the vector of all local coefficients, we have
  \begin{equation}
    \label{eq:51}
    \hat{u}^\ell = \left[
      \begin{array}[c]{c}
        \hat{u}^1\\
        \hat{u}^2\\
        \vdots\\
        \hat{u}^{N_\text{el}}
      \end{array} \right]
    = \mathcal{A}\ \hat{u}_g
  \end{equation}
  where $\mathcal{A}$ is a very \emph{sparse} matrix containing $1$ (and also $-1$ in multi-dimension)
\end{frame}

\begin{frame}{Assembly matrix}
  \begin{equation}
    \label{eq:52}
    \hat{u}^\ell = \left[
      \begin{array}[c]{c}
        \hat{u}^1_0\\
        \hat{u}^1_1\\
        \hat{u}^2_0\\
        \hat{u}^2_1\\
        \hat{u}^3_0\\
        \hat{u}^3_1\\
      \end{array} \right]
    =
    \begin{pmatrix}
      1 & & &\\
      & 1 & &\\
      & 1 & &\\
      &  & 1 &\\
      &  & 1 &\\
      &  &  & 1 \\
    \end{pmatrix}
    \
    \left[\begin{array}[c]{c}
        \hat{u}_0\\
        \hat{u}_1\\
        \hat{u}_2\\
        \hat{u}_3\\
      \end{array} \right]
  \end{equation}
  The matrix $\mathcal{A}$ \emph{scatters} $\hat{u}_g$ in the local
  degrees of freedom. The reverse operation is advantageous since it
  will allow us to perform local operations and assemble them in the
  global degrees of freedom. It is given by $\mathcal{A}^T$.
\end{frame}

\begin{frame}{Integral computation}
  Consider now that you have to compute the algebraic $l(\Phi_i) =
  \int_\Omega  \Phi_i(x) f(x)$, and taking $i=1$ in the 1D previous example:
  \begin{equation}
    \label{eq:53}
    \int_\Omega \Phi_1(x)\ f(x)\ dx  = \int_{-1}^1\ \phi^1_1(\xi) f(\delta(\chi^1)) \Big|\frac{d \chi^1}{d \xi}\Big| d\xi + \int_{-1}^1 \phi^2_0(\xi) f(\chi^2) \Big|\frac{d \chi^2}{d \xi}\Big| d\xi
  \end{equation}
  From this we see that we can compute integrals locally on the
  standard element $\Omega^{\text{st}}$ using say a numerical
  quadrature. $\mathcal{A}^T$ will allow us to assemble the global
  expansion from the local expansion. This process is called
  \textbf{global assembly}.
\end{frame}

\begin{frame}{Integral computation}
  Denote now the vector $I_g$ of global integral contributions with $i$-th component
  \begin{equation}
    \label{eq:54}
    I_g[i] = \int_\Omega \Phi_i(x)\ f(x) dx
  \end{equation}
  and $I_\ell$ of local contributions
  \begin{equation}
    \label{eq:55}
    I_\ell = \left[
      \begin{array}[c]{c}
        I^1\\
        I^2\\
        \vdots\\
        I^{N_\text{el}}
      \end{array}\right]
    \text{ where }
    I^e =  \left[
      \begin{array}[c]{c}
        \int_{-1}^1\ \phi_0(\xi) f(\chi^e) \frac{d \chi^e}{d \xi} d\xi\\
        \vdots\\
        \int_{-1}^1\ \phi_{P-1}(\xi) f(\chi^e) \frac{d \chi^e}{d \xi} d\xi
      \end{array}\right]
  \end{equation}
  $I_g$ can then be related to the local elemental vector $I_\ell$ using $\mathcal{A}^T$
  \begin{equation}
    \label{eq:56}
    I_g = \mathcal{A}^T I_\ell
  \end{equation}
  This operation performs the \textbf{summation} of the local contribution into the global expansion
\end{frame}

\begin{frame}[containsverbatim]{DOF table}
  In practice $\mathcal{A}$ is \alert{never assembled} as it is very
  sparse. We use a mapping array also known as Degrees of freedom
  table (DOF table) which maps for each element $e$ its local degrees
  of freedom $i$ to a global degree of freedom $j$
  \begin{lstlisting}
    map[e][i] = j
  \end{lstlisting}

  Hence to compute $I_g$, the assembly process reads
  \begin{lstlisting}[mathescape,texcl]
    for( int e = 0; e < N_el; ++e )
      for( int i = 0; i < P; ++p )
       $I_g$[map[e][i]] = $I_g$[map[e][i]] + $I^e[i]$
  \end{lstlisting}
\end{frame}

\begin{frame}{Bilinear form assembly}
  Now we turn to the assembly of the bilinear form associated to our
  problem, e.g $a(v, u^\delta)\ =\ \int_\Omega\ v\ u^\delta\ ,\  v \in \mathcal{V}^\delta$

  Recall that we have
  \begin{equation}
    \label{eq:59}
    u \stackrel{\text{accurately}}{\approx}  u^{\delta }(x, t ) = \sum_{i=1}^{N_{\text{dof}}}\ \hat{u}_i(t) \Phi_i(x)
  \end{equation}
  and $v$ will take the $\Phi_i, i=1,...N_{\text{dof}}$, and denote
  $S_{\text{el}}$ is the set of element indices that share the degrees
  of freedom $i$ and $j$.

  \begin{equation}
    \label{eq:60}
    \begin{array}[c]{rl}
      a(\Phi_i, u^\delta) & = \int_\Omega\ \Phi_i\ \sum_{j=1}^{\ndof}\ \hat{u}_j\ \Phi_j \\
      & = \int_{\Omega^{\text{st}}} \ \phi^e_\iloc(\xi)\ \Big( \sum_{e\ \in\ S_{\text{el}}}\ \sum_{\jloc=0}^{\nldof}\ \hat{u}_\jloc^e\ \phi_\jloc^e(\xi) \Big) J^e(\xi) \\
      & = \sum_{e\ \in\ S_{\text{el}}}\ \sum_{\jloc=0}^{\nldof}\ \hat{u}_\jloc^e\ \underbrace{\int_{\Omega^{\text{st}}} \ \phi^e_\iloc(\xi)\   \phi_\jloc^e(\xi)  J^e(\xi)}_{\aloc{e}(\phi_\iloc, \phi_\jloc)} 
    \end{array}
  \end{equation}
\end{frame}
\begin{frame}{Algebraic form}
  Algebraically we have
  \begin{equation}
    \label{eq:61}
    a(\Phi_i, u^\delta) = \underbrace{A(i,:)}_{(A_{ij})_{i,j=1}^{N_{\text{dof}}} } \underbrace{\hat{u}}_{\hat{u}_{j=1}^{N_\text{dof}}},\quad i = 1,...,\ndof
  \end{equation}
  where
  \begin{equation}
    \label{eq:62}
    A_{ij} = \sum_{e\ \in\ S_{\text{el}}}\ \sum_{\jloc=0}^{\nldof} \aloc{e}(\iloc,\jloc),\ i = \text{map}(e,\iloc), j = \text{map}(e,\jloc)
  \end{equation}
\end{frame}

\begin{frame}{Remarks}
  Some remarks from there are in order:
  \begin{itemize}
  \item We still need to do the polynomials constructions and
    operations on $\Omega^{st}$
  \item $A$ is a sparse matrix, the sparsity depends on the polynomial
    order and the number of elements, efficient storage and operations
    are in order. We shall see later in the course how to deal with
    the matrix format.
  \end{itemize}
\end{frame}

\begin{frame}[containsverbatim]{Code to assemble}
  \begin{lstlisting}[mathescape,texcl]
    for( int e = 1; e < $\nel$; ++e )
    {
      // local assembly
      for( int $\iloc$ = 0; $\iloc$ < $\nldof$; ++$\iloc$ )
        for( int $\jloc$ = 0; $\jloc$ < $\nldof$; ++$\jloc$ )
         // compute $\aloc{e}( \iloc, \jloc ) = \int_{\Omega^{\text{st}}} \ \phi_\iloc(\xi)\   \phi_\jloc^e(\xi)  J^e(\xi)$
         // $\aloc{e}$ is a dense matrix holding the 
         // _all_ contributions from element $e$

      // global assembly      
      for( int $\iloc$ = 0; $\iloc$ < $\nldof$; ++$\iloc$ )
        for( int $\jloc$ = 0; $\jloc$ < $\nldof$; ++$\jloc$ )
         // $i = \mathrm{map}[e][\iloc]$
         // $j = \mathrm{map}[e][\iloc]$
         // add contribution from element $e$ to $A$, 
         // i.e. $A(i,j) += \aloc{e}( \iloc, \jloc )$
    } // e
  \end{lstlisting}
\end{frame}

\begin{frame}{Performance remarks}
  
  \begin{itemize}
  \item The local assembly stage is clearly a computing intensive block. 
  \item The global one is not clearly an hindrance to performance, but
    actually it is: this stage requires to have efficient memory
    handling and avoid doing any memory allocation in the matrix $A$ for new entries $(i,j)$
  \end{itemize}
  \begin{block}{Free software linear algebra libraries}
    Libraries such as
    Petsc\footnote{\url{http://www-unix.mcs.anl.gov/petsc/petsc-as/}}
    and Trilinos\footnote{\url{http://trilinos.sandia.gov}} offer in
    their API\footnote{Application Programming Interface} routines to ensure proper global assembly stage in
    squential and parallel (e.g. providing the graph of the matrix.)
  \end{block}
\end{frame}

\begin{frame}[containsverbatim]{C++}
  So far we haven't used much C++. The only concept that we have seen
  is the notion of DOF table. This would actually be a 2D array table
  (e.g. \lstinline{Boost::multi_array}\footnote{\url{http://www.boost.org/libs/multi_array/doc/user.html}}). It could be also a map
  (e.g. \lstinline{std::map}\footnote{\url{http://www.cplusplus.com/reference/stl/map/}})
  
  \begin{lstlisting}[mathescape]
    #include <boost/multi_array.hpp>
    boost::multi_array<int,2> dof( boost::extents[$\nel$][$\nldof$] );

    #include <map>
    std::map<std::pair<int,int>, int> dof;
  \end{lstlisting}
  The former is much more efficient, but will fail in a true hp setting. 
\end{frame}
\subsection{Geom. mapping}
\label{sec:geom.-mapping}
\begin{frame}{Geometric mapping}
  We have seen how we are going to use the geometric mapping between a
  reference element/domain $\Omst \subset \setR{n}$ and a real $\Om{e} \subset \setR{n}$.
  We shall use the following notations  and formulas
  \begin{equation}
    \label{eq:66}
    \chi^e( \xi )\ =\ \sum_{i=0}^{\ngdof} G_i\ \phi_i (\xi)\ =\ G\ \phi(\xi) 
  \end{equation}
  
  Denote $K^e(\xi)$ its gradient in $\xi$ ($n\times n$ matrix):
  \begin{equation}
    \label{eq:67}
    K^e( \xi )\ =\ \nabla\ \chi^e (\xi )\ =\ \sum_{i=0}^{\ngdof}\ G_i\ \nabla \phi_i (\xi)
  \end{equation}
  and $B^e(\xi)$ the inverse transposed in $\xi$ ($n\times n$ matrix)
  \begin{equation}
    \label{eq:68}
    B^e( \xi )\ =\ \Big(K^e(\xi)\Big)^{-T}
  \end{equation}
  And finally denote $J^e(\xi)$ the jacobian in $\xi$
  \begin{equation}
    \label{eq:69}
    J^e(\xi)\ =\ |\det( K^e(\xi) )|
  \end{equation}
\end{frame}

\begin{frame}{Change of variables in integrals}
    \begin{equation}
      \label{eq:71}
      \int_{\Om{e}} \ f\ dx\ =\ \int_{\Omst} f( (\chi^e)^{-1}(\xi) ) J^e( \xi )\ d \xi
  \end{equation}
  \begin{equation}
    \label{eq:70}
    \int_{\Om{e}}\ \nabla f\ dx\ =\ \int_{\Omst} \Big(\nabla^{\text{st}} f( (\chi^e)^{-1}(\xi) ) B^e(\xi)\Big) J^e( \xi )\ d \xi
  \end{equation}
  \begin{equation}
    \label{eq:65}
    \int_{\partial \Om{e}}\ f( x )\ \mathbf{n}(x)\ dx = \int_{\partial \Omst} f( (\chi^e)^{-1}(\xi) )\  \| B^e(\xi) \ \mathbf{n^{\text{st}}}(\xi) \|\ J^e( \xi )\ d \xi
  \end{equation}
  where $\mathbf{n}(x)$ is the outward normal to $\partial \Om{e}$
  evaluated in $x \in \partial \Om{e}$, and $\mathbf{n}^{\text{st}}(\xi)$
  the outward normal to $\Omst$ evaluated in $\xi \in \partial \Omst$.
\end{frame}

\begin{frame}{Next steps}
  We now turn to the constructions and operations in $\Omst$
\end{frame}


\end{document}


%%% Local Variables:
%%% mode: latex
%%% TeX-master: "scicomp-galerkin-print"
%%% TeX-PDF-mode: t
%%% TeX-parse-self: t
%%% x-symbol-8bits: nil
%%% TeX-auto-regexp-list: TeX-auto-full-regexp-list
%%% ispell-local-dictionary: "american"
%%% End:

