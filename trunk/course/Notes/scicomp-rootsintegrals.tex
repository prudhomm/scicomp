%
%
% SUMMARY:
% USAGE:
%
% AUTHOR:       Christophe Prud'homme
% ORG:          Christophe Prud'homme
% E-MAIL:       prudhomm@zion
%
% ORIG-DATE:  7-Apr-04 at 16:48:32
% LAST-MOD:  7-Apr-04 at 23:07:19 by Christophe Prud'homme
%
% DESCRIPTION:
% DESCRIP-END.

\date{January 7 2008}

\begin{document}

% For every picture that defines or uses external nodes, you'll have
% to apply the 'remember picture' style. To avoid some typing, we'll
% apply the style to all pictures.
\tikzstyle{every picture}+=[remember picture]
\tikzstyle{na} = [baseline=-.5ex]

%By default all math in TikZ nodes are set in inline mode. Change this to
% displaystyle so that we don't get small fractions.
\everymath{\displaystyle}

\lecture[3]{Roots finding and Numerical Integration}{integ}
\subtitle{}

\begin{frame}
  \maketitle
\end{frame}

\begin{frame}
  \tableofcontents
\end{frame}

\section{Fixed Point}
\label{sec:fixed-point}

\subsection{Framework}
\label{sec:framework}


\begin{frame}{}
  
\end{frame}


\subsection{Newton Raphson}
\label{sec:newton-raphson}

\begin{frame}{}
  
\end{frame}


\subsection{Multiple Roots}
\label{sec:multiple-roots}

\begin{frame}{}
  
\end{frame}


\subsection{Secant Method}
\label{sec:secant-method}

\begin{frame}{}
  
\end{frame}

\subsection{Systems of Non-linear equations}
\label{sec:systems-non-linear}

\begin{frame}{}
  
\end{frame}

\subsection{Continuation methods}
\label{sec:continuation-methods}

\begin{frame}{}
  
\end{frame}


\section{Solutions via Minimization}
\label{sec:solut-via-minim}

\subsection{Gradient}
\label{sec:gradient}


\begin{frame}{}
  
\end{frame}

\subsection{Conjuguate Gradient}
\label{sec:conjuguate-gradient}


\begin{frame}{}
  
\end{frame}


\section{Numerical integration}
\label{sec:numer-integr}

\subsection{Standard rules}
\label{sec:standard-rules}

\begin{frame}{}
  
\end{frame}

\subsection{Jacobi polynomials}
\label{sec:jacobi-polynomials}

\begin{frame}{}
  
\end{frame}

\subsection{Gaussian quadratures}
\label{sec:gaussian-quadratures}

\begin{frame}{}
  
\end{frame}

\subsection{multi-dimension integration over infinite interval}
\label{sec:multi-dimens-integr}

\begin{frame}{}
  \begin{itemize}
  \item Laguerre
  \item Hermite
  \end{itemize}
\end{frame}


\subsection{multi-dimension integration}
\label{sec:multi-dimens-integr}

\begin{frame}{}
  
\end{frame}

\section{Programming}
\label{sec:programming}

\subsection{MPI Reduce operations}
\label{sec:mpi-reduce-oper}

\begin{frame}{}
  
\end{frame}

\subsection{C++}
\label{sec:c++}

\begin{frame}{Passing functions}
  \begin{itemize}
  \item pointers
  \item functors
  \item Boost.Function
  \item Boost.Lambda: anonymous functions
  \end{itemize}
\end{frame}

\subsection{Libraries}
\label{sec:libraries}

\begin{frame}{}
  
\end{frame}


\end{document}


%%% Local Variables: 
%%% mode: latex
%%% TeX-master: "scicomp-approx-print"
%%% TeX-PDF-mode: t
%%% TeX-parse-self: t
%%% x-symbol-8bits: nil
%%% TeX-auto-regexp-list: TeX-auto-full-regexp-list
%%% ispell-local-dictionary: "american"
%%% End: 


