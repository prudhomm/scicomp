\documentclass[12pt]{beamer}

\usetheme{Zaragoza}
%\usetheme[height=3.5\baselineskip]{Zaragoza}%
%\usetheme[height=1.9cm]{Zaragoza}%
%\usetheme[sidebar, height=1.9cm, right]{Zaragoza}
%\usetheme[sidebar, height=1.6cm, left]{Zaragoza}
%\usetheme[showsections, height=1.3cm]{Zaragoza}

\title{Zaragoza Beamer Theme}
\author{J.I. Montijano \& T. Mu\~{n}oz}
\institute[Univ. Zaragoza]{University of Zaragoza \\  Spain}
\date{\today}

%\logo{\includegraphics[width=1.3cm]{unizar.jpg}}

\begin{document}

\frame{\titlepage}

\section[]{}

\frame[t]{\tableofcontents}

\section[Use of the theme]{Use of the Zaragoza beamer theme}

\parskip 5pt

\frame[c]{\frametitle{Theme Zaragoza}

The theme is used in the standard way

\begin{itemize}
\item \texttt{ \structure{ $\backslash$usetheme[``options'']\{Zaragoza\}  }}
\end{itemize}
}

\subsection[Theme options]{Zaragoza beamer theme options}

\frame[t]{%\frametitle{Theme Zaragoza}\framesubtitle{Options}

Several options are available:
\structure{
\begin{enumerate}
\item \texttt{showsections}
\item \texttt{hidesections} \quad (default)
\item \texttt{height = <dimension>}
\item \texttt{sidebar}
\begin{itemize}
\item \texttt{width = <dimension>}
\item \texttt{left}  \quad (default if sidebar is set)
\item \texttt{right}
\end{itemize}
\end{enumerate}
}

Example:

\texttt{ \structure{ $\backslash$usetheme[sidebar,left,width=1.8 cm]\{Zaragoza\}  }}
}

\section[General color]{Control of the general color}

\frame{\frametitle{How to change the color}
The following macro is provided to change the color of the frames:

\begin{block}{Changing the color}
\texttt{ $\backslash$tone\{``color''\} }
\end{block}

For example the orders (written outside the frames)

\qquad \texttt{$\backslash$definecolor\{mygreen\}\{rgb\}\{0.20, 0.40, 0.20\} }\\
\qquad \texttt{$\backslash$tone\{mygreen\}}

change the color of the following frames to myblue

}

%\definecolor{myblue}{rgb}{0.42, 0.62, 0.89}
%\tone{myblue}
\definecolor{mygreen}{rgb}{0.20, 0.40, 0.20}
\tone{mygreen}

\section[Other title page]{Other title page}

\title{Part II}
\frame{\titlepage}

\section[Customization]{Customizing the theme}

\setbeamertemplate{background}{}

\begin{frame}
\frametitle{Customizing the theme}

If you don't like the background, you can ``remove'' it with

\texttt{ $\backslash$setbeamertemplate\{background\}\{\} }

and you can recover it later with

\texttt{ $\backslash$setbeamertemplate\{background\}[cylinder theme] }

\end{frame}

\setbeamertemplate{footline}{}
\setbeamertemplate{background}[cylinder theme]

\begin{frame}
\frametitle{Customizing the theme}

If you don't like the footline, you can ``remove'' it with

\texttt{ $\backslash$setbeamertemplate\{footline\}\{\} }

and you can recover it later with

\texttt{ $\backslash$setbeamertemplate\{footline\}[cylinder theme] }

\end{frame}

\setbeamertemplate{footline}[cylinder theme]

\begin{frame}[plain]

The option \texttt{[plain]} added to the frame
eliminates almost everything in the frame

\end{frame}

\end{document}

