\documentclass{beamer}
\usepackage{clock}

\usetheme{Madrid}


\title[A clock in beamer]{An example of using Clock in beamer}
\author{Luis R\'{a}ndez \& Juan I. Montijano}
\institute[IUMA]{IUMA \\ Universidad de Zaragoza }
\date{\today}


\begin{document}


\frame{\titlepage}

%\frame{\tableofcontents}


\begin{frame}[t]

\frametitle{Introduction}

\initclock  % This must be used one time to initialize the clock

We are putting here a clock with the macro  \quad
\structure{\texttt{$\backslash$insertclock}}

\medskip

The default settings are used:  \insertclock

\medskip

Before that, the macro \quad  \structure{\texttt{$\backslash$initclock}}
\quad must have been used (only one time !).

\end{frame}

% This is an example for inserting
% the clock at the place reserved to the date, the foot line in the
% case of Madrid theme.
%  Colors are adjusted according the theme.
%
\date{\insertclock}
\usebeamercolor{palette primary}
\clockbg{bg}
\clockbdcolor{bg}
\clockfontcolor{fg}


\begin{frame}

Here we have inserted the clock at the place reserved to the date, the foot line in the case of Madrid theme.

\medskip

To do that, we have redefined the date contents with the order
\medskip

\centerline{\structure{\texttt{$\backslash$date\{$\backslash$insertclock\}}} }

\medskip

Colors are adjusted according to the theme.

\medskip

Look at the foot!, the clock must be running.
\end{frame}

\date{\inserttogglebutton{}\insertclock\insertresetbutton{}}

\begin{frame}

Now, in addition to the clock, we also insert a button to toggle between clock and stopwatch (left button), and a button to reset the stopwatch to zero (right button).

\medskip

To do that, we have redefined the date with the order

\medskip

\structure{
\texttt{$\backslash$date\{$\backslash$inserttogglebutton\{\}%
$\backslash$insertclock$\backslash$insertresetbutton\{\}\}}
}

\medskip

The clock must be running at the foot. Click at the left button, see what happens,
and click at the right button and see the result.
\end{frame}


%
% Here we change the colors and size of the clock
%
\definecolor{myred}{rgb}{0.8,0,0} % rojo
\definecolor{mycolor}{rgb}{0.80,0.50,0.20} %  darkorangere

\clockbg{mycolor}
\clockbdcolor{myred}
\clockfontcolor{black}
\clockheight{0.20cm}
\clockwidth{2.0cm}
\clockfontsize{5pt}


\begin{frame}[fragile]
Here we have just changed the colors and the size of the clock using the next sentences

\begin{verbatim}
\definecolor{myred}{rgb}{0.8,0,0}
\definecolor{mycolor}{rgb}{0.80,0.50,0.20}
\clockbg{mycolor}
\clockbdcolor{myred}
\clockfontcolor{black}
\clockheight{0.20cm}
\clockwidth{2.0cm}
\clockfontsize{5pt}
\end{verbatim}


\end{frame}

\begin{frame}

Next, we insert a toggle button, with a text:

\medskip

\clocktogglecolor{red}
\quad \inserttogglebutton{\bf Click here to toggle clock-stopwatch}

\medskip

Here the clock \quad \insertclock

Here just a stopwatch \quad \insertcrono

Here just the current time \quad \inserttime

Here just the current date \quad \insertcurrentdate

\end{frame}


\end{document}

